\documentclass{article}
\usepackage[utf8]{inputenc}
\usepackage{graphicx}
	\DeclareGraphicsExtensions{.png, .jpeg}
\usepackage[top=1in, bottom=1in, left=1in, right=1in]{geometry}

\title{JSTCC: A simple C Compiler \\ CS660: Compiler Construction}
\author{Janelle Blankenburg, Shubham Gogna, Terence Henriod}
\date{\today}

\begin{document}

\clearpage            % All of
\maketitle            % this,
\thispagestyle{empty} % removes the page number from the title page

\begin{abstract}
\noindent Over the course of the Fall 2015 semester we implemented a basic
compiler for the C programming language. The compiler is composed of a front-end
and a back-end, which interface through a psuedo-assembly language of our own
design (referred to as ``3 Address Code" or ``3AC"). The compiler translates
ANSI C to MIPS assembly that can be run using the Mips Assembly and Runtime 
Simulator (MARS).
\end{abstract}

\newpage
\section{Compiler Construction Course}
\noindent The goal of the course was to implement a basic, ANSI C compiler, oriented
toward learning compiler concepts and not necessarily something ``production
ready."\\

\noindent Over the course of the semester, a compiler that uses a front-end $\rightarrow$
intermediate representation $\rightarrow$ machine code methodology was used.
The compiler was constructed in stages, such that an end-to-end compilation
was not achieved until the end, requiring better initial design of components
than might have been required otherwise.\\

\noindent The product of the course took the form of \texttt{jstcc.py}, a python script
embodying the compiler. The compiler was named for the members of the group:
the \textbf{J}anelle \textbf{S}hubham \textbf{T}erence \textbf{C} 
\textbf{C}ompiler.


\section{Implementation}
\subsection{Overview}
\noindent The compiler was implemented in the Python3 programming language in order to
emphasize compiler concepts and code readability. A few third party libraries
are used, but the majority of the code and design is our own.\\

\noindent The team consisted of two graduate students (Janelle, Terence) and one
undergraduate (Shubham), and completed the graduate level requirements of the
course.\\

\noindent The compiler is composed of the following steps or modules, which will be 
described in detail later:
  \begin{enumerate}
  \item Symbol Table
  \item Scanner/Tokenizer
  \item Parser/Syntax-Directed Translator
  \item Abstract Syntax Tree (AST)
  \item Three Address Code (3AC)
  \item Machine Code (MIPS Assembly language)
  \end{enumerate}


\subsection{Dependencies}
\noindent The following $3^{rd}$ party libraries were used:
  \begin{itemize}
  \item \textbf{Python3 Interpreter}: Necessary for running Python code.

  \item \textbf{bintrees}: A pure python balanced-tree data structure library.
                            Used to meet the requirement of using balanced
                            trees in the symbol table.

  \item \textbf{PLY}: Python Lex-Yacc. A pure python, tried and true Lex and 
                      Yacc implementation. these were used to implement the
                      scanner/tokenizer and the parser or syntax-directed
                      translator.

  \item \textbf{pylru}: A python LRU cache implementation used to implement (in
                        part) register management in assembler output.
  \end{itemize}

\subsection{Usage}
\subsubsection{JSTCC}
\noindent In order to run our compiler, you must have the above dependencies installed.
Then you can run the compiler from the command line with a command like:\\
\texttt{
\$ python relative/path/to/JST/project/bin/jstcc.py <flags> <c program file>
}\\

\noindent JSTCC has several command line options to produce or add various items to its
output:
  \begin{itemize}
  \item \texttt{-h}, \texttt{--help}: Displays usage information for the
        program.
  \item \texttt{-o}, \texttt{--outfile}: Designate a \texttt{.asm} file to place
        the mips output in (default: \texttt{stdout}).
  \item \texttt{-sym}, \texttt{--symtable} Add a dump of the symbol table to the
        output.
  \item \texttt{-s}, \texttt{--scandebug}: Add printing of tokens and source
        code to the output during tokenization.
  \item \texttt{-p}, \texttt{--parsedebug}: Add printing of productions and
        source code to the output during parsing.
  \item \texttt{-ast}, \texttt{--astree}: Add the textual representation of the
        tree in the GraphViz language to the output.
  \item \texttt{-tac}, \texttt{--threeac}: Add 3AC and source code to the
        output.
  \item \texttt{-mips}, \texttt{--mips}: Add 3AC and source code to the
        generated \texttt{.asm} file.
  \item \texttt{-w}, \texttt{--warnings}: Activate printing of compile-time
        warnings to the \texttt{stdout}.
  \end{itemize}

\subsubsection{Web-Site}
\noindent Our website features presentations of our project through its various stages,
including presentation forms of our most important test cases and their output,
output including informative intermediate representations, MIPS assembly, and
the output of running MIPS programs in MARS.\\

\noindent To access the web-page statically, simply open the \texttt{index.html} file with
a web browser (the operating system often has an association between this type
of file and the web-browser, so opening the file should work).


\subsection{Symbol Table}
\noindent The symbol table was used to contain information about all of the declared items
in the program (variables, functions).\\

\noindent In the beginning of the project, the symbol table played a central role in
the organization of the compiler's information. However, as the project went on,
the symbol table's use became reduced to almost just tracking memory usage and
function signatures.\\

\subsection{Scanning and Parsing}
\noindent PLY was used to implement the Scanner and Parser. This allowed us to implement
this items in an easily extensible way.\\

\noindent For the scanner, much of the difficulty came in finding the correct regular
expressions to describe the tokens properly, but in the end this was very 
doable.\\

\noindent As for the parser, it was a harsh realization that it played an extremely large
role in the project, had we realized this sooner, the project would have gone
much better. Getting accustomed to thinking in the way of an LR parser was
also difficult, but once we did, things progressed. This is an area of the
project that we think we could really do much better with had we known what it
would entail (``If I knew what I know now when I was younger").\\

\subsection{Abstract Syntax Tree}
\noindent The AST was a real turning point for the group in terms of the project. This
was the point when we started to glimpse ``the big picture."\\

\noindent Designing the tree nodes was straightforward for the most part, but when it
came to designing nodes like the Symbol node, there was really no good way to
design it properly without knowing what we would learn later in the semester.
Getting a good Symbol node and an excellent type system would be the areas that
it would have been most effective to re-implement given the chance.\\

\noindent Our AST used relatively few nodes, and we aimed to keep a set of nodes that 
eschewed anything extraneous and would not result in intermediate code, but not
have a set so minimal that the nodes were no longer modular. Further, we feel
our design would be easily extensible given that the nodes were designed to
interface with one another in such a way that they would not block another
node's functionality.

\subsection{Three Address Code}
\noindent Our psuedo-assembly 3AC was closely modeled after MIPS. In retrospect, it 
would have been nice to build something more general, but since we were 
building a machine to translate to MIPS, this was acceptable.\\

\noindent The 3AC was designed to accept registers, literals, and addresses as parameters,
as opposed to types for arguments as others may have done, but we felt this
greatly eased the transition to actual assembler code. The drawback, of course,
is difficulty in use of the co-processor, but it certainly would not be
prohibitive, and floating point operations could be implemented with little
more effort than had we taken the alternative approach.\\

A note on implementation: we used a single, generic 3AC instruction class; if we
could, we would revise this to use a unique class for each instruction, possibly
inheriting from relevant base-classes to create categories of related
instructions.

\subsection{Assembler: MIPS}
\noindent The implementation of the final translation to MIPS was both a relief and a
pain. We finally were able to see our project come to fruition (since the
compiler was built in such a way that it was difficult to know exactly how
things would turn out, building in a cross-sectional instead of longitudinal
way).\\

\noindent In order to implement the MIPS translation, several supporting frameworks had
to be created. The first was a register use table, to help manage the mapping
of psuedo-registers (used prolifically in the 3AC) to the limited physical
registers required for MIPS programming. The other was The development of
MIPS macros that could be inserted into any program, greatly reducing the
footprint of any generated code.\\

\noindent The most difficult part of MIPS translation was the design and implementation
of the stack/activation frame. This was complex and very error prone, partly
due to the confusion that comes with the fact that \textit{stacks grow down}.
Also difficult was determining how to separate responsibilities of 
\textit{caller}s and \textit{callee}s. In the end, the result was as follows:
  \begin{enumerate}
  \item The caller pushes registers and a copy of the register-spill memory
        whose values need to be saved onto the stack.

  \item The caller then copies/pushes the values of any function call arguments
        onto the stack.

  \item The caller executes a jump-and-link instruction to the callee.

  \item The callee then pushes memory onto the stack for all of its local
        variables.

  \item The callee executes its body.

  \item Upon completion of the function body or a return statement, the callee
        stores any relevant return value to the designated return register
        and de-allocates its local variable and argument memory.

  \item The callee jumps back to the caller.

  \item The caller then recovers the spill memory and register values.

  \item Finally, the caller copies the return value from the designated return
        register for use.
  \end{enumerate}

\noindent A triumph in our register use was detecting points in the program when we no
longer needed to maintain mappings for pseudo-registers to physical ones. This
greatly economized our use of spill memory, often eliminating the need to spill
registers altogether! This was very good considering that we limited ourselves
to roughly half of the available MIPS registers


\section{Compiler Capabilities}
\noindent The following items were requirements of the course:
  \begin{enumerate}
  \item Basic declarations and assignments.

  \item Simple arithmetic operations.

  \item Conditional statements (\texttt{if}, \texttt{else if}, \texttt{else}).

  \item \texttt{while} loops.

  \item One, two, and three dimensional arrays.

  \item Basic function calls with simple parameters.
  \end{enumerate}

\noindent Our compiler supports the following additional features:
  \begin{enumerate}
  \item \texttt{for} and \texttt{do... while} loops.

  \item Arbitrarily high dimensional arrays.

  \item Recursive function calls.

  \item Functions that accept arrays as parameters.

  \item Acceptance of ``fragmented strings", where string fragments such as
        \texttt{"hello" "world" "!"} are combined into a single string literal.
  \end{enumerate}

\hfill\newline
\noindent Additionally, the following optimizations are supported:
  \begin{enumerate}
  \item Constant folding.
  \end{enumerate}


\section{Future Work}
\noindent The following are features that, if given more time, are features that we would
like to or be able to implement. There are simple ones that would only take days
to implement, and complex ones that would require more effort.

\subsection{Simple}
  \begin{itemize}
  \item \textit{Flesh out floating point operations}. We made a start, but did
        not have time to see it through given that it was not a requirement
        of the course.

  \item \textit{Make use of floating point and string literals}. We have a good
        concept of how to do this, but since it was not a requirement we did
        not quite make it there.

  \item \textit{More minor optimizations}. We would like to implement other
        simple optimizations such as replacing multiplications or divisions
        by two with bit-shifting operations.

  \item \textit{Implement pointers}. We were very close in thought and effort
        to accomplishing this, but again, since it was not a requirement, we
        refocused our efforts elsewhere. 
  \end{itemize}

\subsection{Complex}
  \begin{itemize}
  \item \textit{Enhanced error reporting}.

  \item \textit{Flesh out all basic language features}. It would have been good
        to implement ternary operators and \texttt{switch} statements, but given
        their redundancy to other language features, they would be more for 
        completeness and a bookkeeping exercise than an actual enhancement.

  \item \texttt{struct}s. These are the next step in a complete language and
        implementing them would be very informative about memory management
        concepts.

  \item \texttt{typedef}s. For both the convenience and the ``intellectual
        street cred."

  \item \textit{Function pointers}. To truly master the language, create a
        language capable of true generality, and to better understand code
        manipulation. 
  \end{itemize}

\section{Recommendations}
\noindent We would first recommend to anyone building a compiler in the future to make
use of a unit test framework. Unit testing as much of our code as possible,
even if it was just making sure that objects resulting from processes could
be stringified in a predictable manner or to evaluate if fatal errors occurred
in the code was instrumental in our progress. Being able to test things in an
automated manner saved many hours of strained eyes and tired fingers.\\

\noindent Second, we would recommend that compiler-implementing-students look to
real-world examples as much as possible. We drew inspiration from a library
called \textit{pycparser}, a python C parsing library, for most of the front-
end. This was helpful, but in retrospect, we wish we would have looked at an
established intermediate language such as \textit{LLVM} to model our 3AC after.
Also it may have provided the benefit of having an already-implemented back-end
to compare against.

\end{document}
