\title{CS 677: Assignment 1}
\author{Terence Henriod}
\date{\today}

\documentclass[11pts]{article}

%\usepackage{qtree}
\usepackage{amsmath}
\usepackage{amssymb}
\usepackage{verbatim}
\usepackage[right=1in,top=1in,left=1in,bottom=1in]{geometry}



\newcommand{\BigO}[1]{\ensuremath{\operatorname{O}\bigl(#1\bigr)}}
\newcommand{\BigTheta}[1]{\ensuremath{\operatorname{\Theta}\bigl(#1\bigr)}}

%\title{CS 677 Homework \\ Assignment 1}
%\date{February 12, 2014}
%\author{Terence Henriod}

\setlength{\topmargin}{-1cm}
\setlength{\oddsidemargin}{0in}
\setlength{\textwidth}{6.5in}
\setlength{\textheight}{8.3in}


%%Currently default settings for indentation and symbols.
%%Try these by uncommenting this block!!!
%%Redefine the first level symbols
%\renewcommand{\theenumi}{\fnsymbol{enumi}-}
%\renewcommand{\labelenumi}{\theenumi}
%
%%Redefine the second level symbols
%\renewcommand{\theenumii}{\alph{enumii})}
%\renewcommand{\labelenumii}{\theenumii}
%
%%Redefine the third level symbols
%\renewcommand{\theenumiii}{\roman{enumiii}.}
%\renewcommand{\labelenumiii}{\theenumiii}
%
%%Options for redefining levels


%\arabic
%\alph 
%\Alph
%\roman
%\Roman
%\fnsymbol
%This ^^^ is all you need to change!!

\begin{document}

\maketitle

\begin{abstract}
In this assignment, the topics of growth order functions and asymptotic 
notations are explored.
\end{abstract}
\newpage

\begin{enumerate}
%================= Problem 1 ===================================================
\item Arrange the following list of functions in ascending
order of growth rate. That is, if function g(n) immediately follows function
f(n) in your list, then f(n) should be O(g(n)).

  \begin{enumerate}
  \item $f_1(n) = n^2.5$

  \item $f_2(n) = \sqrt{2n}$

  \item $f_3(n) = n^3 + 10$

  \item $f_4(n) = 10^n$

  \item $f_5(n) = 100^n$

  \item $f_6(n) = n log n$
  \end{enumerate}
    
  \textbf{Solution:} In general, we can order these terms by intuitively 
  seeking the highest order term in each $f_i(n)$. Should confusion arise, 
  techniques such as L\^Hopital's rule can be used to compare the
  order of functions.

  \begin{align*}
  f_1(n) &= n^{2.5}     \Rightarrow \BigO{n^{2.5}},   \\
  f_2(n) &= \sqrt{2n}   \Rightarrow \BigO{\sqrt{n}},    \\
  f_3(n) &= n^3 + 10    \Rightarrow \BigO{n^3},         \\
  f_4(n) &= 10^n        \Rightarrow \BigO{10^n},        \\
  f_5(n) &= 100^n       \Rightarrow \BigO{100^n},       \\
  f_6(n) &= n \log n    \Rightarrow \BigO{n \log n}
  \end{align*}

  Thus the correct ordering (lowest order to highest) is:
  \begin{align*}
  f_2(n) &= \sqrt{2n}, & f_6(n) &= n \log n,  & f_1(n) &= n^{2.5},     \\
  f_3(n) &= n^3 + 10,  & f_4(n) &= 10^{n},    & f_5(n) &= 100^{n}
  \end{align*}
  \newpage

%================= Problem 2 ===================================================
\item Using mathematical induction, show that the following relations are true
for every $n\ge1$:

  \begin {enumerate}
  %-----------------   A   -----------------------------------------------------
  \item $ \sum_{i=1}^{n}{(-1)^{i+1}i^2} = \frac{(-1)^{n+1}n(n+1)}{2} $ \\

  \textbf{Solution:} When using induction, we must prove a base case,
  assume that the equation holds true for an arbitrary $n$, and then prove
  that the equation holds for the arbritrary $n+1$ value. The three parts
  combine to prove the equation for all $n\ge1$. The process becomes easier
  if the assumption is used to simplify the expression in the inductive step.\\

  \textit{Basis Case}\\
  Let $n=1$:
  \begin{align*}
  \sum_{i=1}^{(1)}{(-1)^{i+1}i^2} &= \frac{(-1)^{(1)+1}(1)((1)+1)}{2} \\
                (-1)^{(1)+1}(1)^2 &= \frac{(-1)^{2}(1)(2)}{2} \\
                       (-1)^{2}*1 &= \frac{1 * (1)(2)}{2} \\
                              1*1 &= \frac{2}{2} \\
                                1 &= 1
  \end{align*}

  \textit{Assumption}\\
  Assume that:
  \begin{align*}
  \sum_{i=1}^{n}{(-1)^{i+1}i^2} &= \frac{(-1)^{n+1}n(n+1)}{2}
  \end{align*}

  \newpage % !!!!!!!!!!!!!!!!!!!!!!!!!!!!!!!!!!!!!!!!!!!!!!!!!!!!!!!!!!!!!!!!!!!
  \textit{To Show}\\
  To prove the formula inductively, let $n=n+1$:
  \begin{align*}
    \sum_{i=1}^{n+1}{(-1)^{i+1}i^2} &= \frac{(-1)^{(n+1)+1}(n+1)((n+1)+1)}{2} \\
    \sum_{i=n+1}^{n+1}{(-1)^{i+1}i^2}+\sum_{i=1}^{n+1}{(-1)^{i+1}i^2} &= 
         \frac{(-1)^{n+2}(n+1)(n+2)}{2}   \\
    (-1)^{n+2}(n+1)^{2}+\sum_{i=1}^{n+1}{(-1)^{i+1}i^2} &= 
         \frac{(-1)^{n+2}(n^{2}+3n+2)}{2} \\
    (-1)^{n+2}(n+1)^{2}+\sum_{i=1}^{n+1}{(-1)^{i+1}i^2} &= 
         \frac{(-1)^{n+2}(n^{2}+3n+2)}{2} \\
    (-1)^{n+2}(n+1)^{2}+\sum_{i=1}^{n+1}{(-1)^{i+1}i^2} &= 
         \frac{(-1)^{n+2}(2n^{2}+4n+2)}{2}-\frac{(-1)^{n+2}(n^{2}+n)}{2} \\
    (-1)^{n + 2} (n + 1)^{2} + \sum_{i = 1}^{n + 1}{(-1)^{i + 1}i^2} &= 
         \frac{(-1)^{n+2}(2n^{2}+4n+2)}{2}-\frac{(-1)(-1)^{n+1}(n^{2}+n)}{2} \\
    (-1)^{n+2}(n^{2}+2n+1)+\sum_{i=1}^{n+1}{(-1)^{i+1}i^2} &= 
         \frac{(-1)^{n+2}(2n^{2}+4n+2)}{2}+\frac{(-1)^{n+1}(n^{2}+n)}{2} \\
    (-1)^{n+2}(n^{2}+2n+1)+\sum_{i=1}^{n+1}{(-1)^{i+1}i^2} &= 
         \frac{(-1)^{n+2}(2n^{2}+4n+2)}{2}+\frac{(-1)^{n+1}n(n+1)}{2} \\
    (-1)^{n+2}(n^{2}+2n+1) &= \frac{(-1)^{n+2}(2n^{2}+4n+2)}{2}
        \text{   By the Assumption}  \\
    (-1)^{n+2}(n^{2}+2n+1) &= (-1)^{n+2}(2n^{2}+2n+1)
  \end{align*}
  \newpage

  %-----------------   B   -----------------------------------------------------
  \item $ \sum_{i=1}^{n}{\frac{1}{(2i-1)(2i+1)}} = \frac{n}{2n+1} $ \\

  \textbf{Solution:} Again, use the three parts of the induction process
  and use the assumption to simplify the process.\\

  \textit{Basis Case}\\
  Let $n = 1$
  \begin{align*}
    \sum_{i=1}^{(1)}{\frac{1}{(2i - 1)(2i + 1)}} &= \frac{(1)}{2(1) + 1} \\
    \frac{1}{(2(1) - 1)(2(1) + 1)} &= \frac{1}{2 + 1} \\
    \frac{1}{(2 - 1)(2 + 1)} &= \frac{1}{3} \\
    \frac{1}{1 * 3} &= \frac{1}{3} \\
    \frac{1}{3} &= \frac{1}{3}
  \end{align*}

  \textit{Assumption} \\
  Assume that:
  \begin{align*}
    \sum_{i=1}^{n}{\frac{1}{(2i-1)(2i+1)}} &= \frac{n}{2n + 1}
  \end{align*}

  \newpage % !!!!!!!!!!!!!!!!!!!!!!!!!!!!!!!!!!!!!!!!!!!!!!!!!!!!!!!!!!!!!!!!!!!
  \textit{To Show}\\
  To prove the formula inductively, let $n = n + 1$
  \begin{align*}
    \sum_{i=1}^{(n+1)}{\frac{1}{(2i-1)(2i+1)}} &= \frac{(n+1)}{2(n + 1) + 1}\\
    \sum_{i=n+1}^{n+1}{\frac{1}{(2i - 1)(2i + 1)}} + 
        \sum_{i=1}^{n}{\frac{1}{(2i - 1)(2i + 1)}} 
        &= \frac{(n+1)}{(2n + 2) + 1}\\
    \frac{1}{((2n + 2) - 1)((2n + 2) + 1)} + 
        \sum_{i=1}^{n}{\frac{1}{(2i-1)(2i+1)}} 
        &= \frac{(n+1)}{2n + 3}\\
    \frac{1}{(2n + 1)(2n + 3)} + \sum_{i=1}^{n}{\frac{1}{(2i-1)(2i+1)}} 
        &= \frac{(n+1)}{2n + 3}\\
    \frac{1}{(2n + 1)(2n + 3)} + \sum_{i=1}^{n}{\frac{1}{(2i-1)(2i+1)}} 
        &= \frac{(n+1)}{2n + 3} * \frac{(2n + 1)}{(2n + 1)}\\
    \frac{1}{(2n + 1)(2n + 3)} + \sum_{i=1}^{n}{\frac{1}{(2i-1)(2i+1)}} 
        &= \frac{(n+1)(2n + 1)}{(2n + 3)(2n + 1)} \\
    \frac{1}{(2n + 1)(2n + 3)} + \sum_{i=1}^{n}{\frac{1}{(2i-1)(2i+1)}} 
        &= \frac{2n^{2} + 3n + 1}{(2n + 3)(2n + 1)} \\
    \frac{1}{(2n + 1)(2n + 3)} + \sum_{i=1}^{n}{\frac{1}{(2i-1)(2i+1)}} 
        &= \frac{2n^{2} + 3n}{(2n + 3)(2n + 1)} + \frac{1}{(2n + 3)(2n + 1)} \\
    \sum_{i=1}^{n}{\frac{1}{(2i-1)(2i+1)}}
        &= \frac{2n^{2} + 3n}{(2n + 3)(2n + 1)} \\
    \sum_{i=1}^{n}{\frac{1}{(2i-1)(2i+1)}}
        &= \frac{(2n + 3) * n}{(2n + 3)(2n + 1)} \\
    \sum_{i=1}^{n}{\frac{1}{(2i-1)(2i+1)}}
        &= \frac{n}{2n + 1}
  \end{align*}

  \end{enumerate}
  \newpage

%================= Problem 3 ===================================================
\item Find the order of growth of the following sums:

  \begin{enumerate}
  %-----------------   A   -----------------------------------------------------
  \item $ \sum_{i=0}^{n-1}{(i^{2}+1)^{2}} $ \\

  \textbf{Solution:} In this problem, the key is to expand the typical element
  formula, distrubute the summation across each element, and finally, find
  the order of the terms of the resulting expression. Some summations will be
  relaced with the known identity: 
  $ \sum_{k=1}^{n}{k^{p}} \approx \frac{1}{p+1}n^{p+1} $
  \begin{align*}
    \sum_{i=0}^{n-1}{(i^{2}+1)^{2}}
        &= ((0)^{2} + 1)^{2} + \sum_{i=1}^{n-1}{(i^{2}+1)^{2}} \\
    &= 0 + \sum_{i=1}^{n-1}{(i^{2}+1)^{2}}\\
    &= \sum_{i=1}^{n-1}{i^{4} + 2i^{2} + 1}\\
    &= \sum_{i=1}^{n-1}{i^{4}} + \sum_{i=1}^{n-1}{2i^{2}} +
        \sum_{i=1}^{n-1}{1}\\
    &= \sum_{i=1}^{n-1}{i^{4}} + 2\sum_{i=1}^{n-1}{i^{2}} + (n - 1)\\
    &= \frac{1}{(4)+1}(n - 1)^{(4)+1} + 2(\frac{1}{(2)+1}(n - 1)^{(2)+1})
        + (n - 1)\\
    &= \frac{1}{5}(n - 1)^{5} + 2(\frac{1}{3}(n - 1)^{3}) + (n - 1)
  \end{align*}
  Thus, by expanding the terms with exponents and ignoring lower order
  terms and multiplicative constants, it is
  easily seen that $ \sum_{i=0}^{n-1}{(i^{2}+1)^{2}} = \BigO{n^{5}} $.
  \newpage

  %-----------------   B   -----------------------------------------------------
  \item $ \sum_{i=2}^{n-1}{\lg{i^{2}}} $ \\

  \textbf{Solution:} This problem is also solved using a similar strategy as
  the previous one. A summation will be replaced with the identity:
  $ \sum_{k=1}^{n}{\lg{n}} \approx n \lg{n} $

  \begin{align*}
    \sum_{i=2}^{n-1}{\lg{i^{2}}} &= \sum_{i=1}^{n-1}{2\lg{i}} - 2\lg{(1)} \\
    &= 2\sum_{i=1}^{n-1}{\lg{i}} - 2(0)\\
    &= 2((n - 1)\lg{(n - 1)}) \\
    &= (2n - 2)\lg{(n - 1)} \\
    &= 2n \lg{(n - 1)} - 2 \lg{(n - 1)}
  \end{align*}
  Again, ignoring lower order terms and multiplicative constants,
  it is seen that $ \sum_{i=2}^{n-1}{\lg{i^{2}}} = \BigO{n \lg{n}} $.
  \end{enumerate}
  \newpage

  %=============== Problem 4 ===================================================
  \item
  For each of the following functions, indicate the class
  $Θ(g(n))$ the function belongs to. Use the simplest $g(n)$ possible in your
  answers.

    \begin{enumerate}
  %-----------------   A   -----------------------------------------------------
      \item $ (n^{2}+1)^{10} $ \\

        \textbf{Solution:} Intuitively, it can be seen that once expanded,
        the leading term of the function will be
        $ (n^{2})^{10} = n^{20} $. Clearly, this term dominates the function's
        behavior, and I propose that $ \forall n_0 \ge 1$ with
        arbitrary constants $ c_1 = 1 $ and $ c_2 = 1024 $
        that $ c_1n^{10} \le (n + 1)^{10} \le c_2n^{10} $.\\

        The lower bound:
        \begin{align*}
          c_1(n^{20}) &\le (n^2 + 1)^{10} \\
          (1)(n^{20}) &\le (n^2)^{10} + \BigO{(n^2)^{9}} \\
               n^{20} &\le n^{20} + \BigO{(n^2)^{9}} \\
                    0 &\le \BigO{n^{18}} 
        \end{align*}
        The upper bound:
        \begin{align*}
          (n^2 + 1)^{10} &= (n^{2})^{10} + \ldots \\
                         &= n^{20} + \BigO{n^{18}} + \ldots \\
                         &= \BigO{n^{20}}
        \end{align*}
        Thus, $ (n^{2}+1)^{10} = \BigTheta{n^{20}} $ by the definition
        of $\Theta$.\\
        \\

  %-----------------   B   -----------------------------------------------------
      \item $ \sqrt{10n^{2}+7n+3} $ \\

        \textbf{Solution:} Due to the radical, this expression is difficult
        to manipulate directly. However, the $10n^2$ term is clearly the
        the dominating term in the radicand. So let
        $ f(x) = \sqrt{10n^2 + \BigO{n}} $ and it follows that:

        \begin{align*}
          \sqrt{10n^2 + \BigTheta{n}} &\approx \sqrt{10n^{2}}  \\
                                      &= \sqrt{10} * \sqrt{n^{2}} \\
                                      &= \sqrt{10} * n \\
                                      &= \BigTheta{n}
        \end{align*}

       Thus, $ \sqrt{10n^{2}+7n+3} = \BigTheta{n}$.\\
       \\

  %-----------------   C   -----------------------------------------------------
      \item $ 2n\lg{(n+2)^{2}}+(n+2)^{2}\lg{\frac{n}{2}} $ \\

        \textbf{Solution:} By simplifying the expressions, we have
        \begin{align*}
          2n\lg{(n+2)^{2}}+(n+2)^{2}\lg{\frac{n}{2}}
              &= 2n(2\lg{n + 2}) + (n^{2} + 4n + 4)(\lg{n} - \lg{2}) \\
          &= 4n\lg{n + 2} + (n^{2} + 4n + 4)(\lg{n} - 1) \\
          &= \BigTheta{n\lg{n}} + \BigTheta{n^{2}} * \BigTheta{\lg{n}} \\
          &= \BigTheta{n^{2}} * \BigTheta{\lg{n}} \\
          &= \BigTheta{n^{2}}
        \end{align*}
        Leaving a highest order term of $n^{2}\lg{n}$.\\
        Thus, $ 2n\lg{(n+2)^{2}}+(n+2)^{2}\lg{\frac{n}{2}}
            = \BigTheta{n^{2}\lg{n}} $.\\
      \\

  %-----------------   D   -----------------------------------------------------
      \item $ 2^{n+1}+3^{n-1} $ \\

        \textbf{Solution:} Although to start, the $2^{n+1}$ is the dominating
        term, $ \forall n \ge 5 $, $ 3^{n-1} $ will be the dominating term: \\

        \begin{center}
        \begin{tabular}{| c || c | c | c | c | c |}
          \hline
          $n$ & 1 & 2 & 3 & 4 & 5 \\
          \hline \hline
          $2^{n+1}$ & 4 & 8 & 16 & 32 & 64 \\
          \hline
          $3^{n-1}$ & 1 & 3 & 9 & 27 & 81 \\
          \hline
        \end{tabular}
        \end{center}
        \newline
        Thus, $ 2^{n+1}+3^{n-1} = \BigTheta{3^n}$.
      \\
    \end{enumerate}
    \newpage

%================= Problem 5 ===================================================
  \item
  For each of the following functions, indicate how much the
  function’s value will change if its argument is increased fourfold:

    \begin{enumerate}
  %-----------------   A   -----------------------------------------------------
    \item $ f(n)=\log_2{n} $ \\

    \textbf{Solution:} In this case, I will replace the $n$ in $f(n)$ with
    $4n$ to create $f'(n)$ and compute the \emph{difference} of the two
    functions.\\
    Let $f'(n) = \log_2{4n}$

    \begin{align*}
    f'(4n) &= \log_2{4n}            \\
           &= \log_2{4} + \log_2{n} \\
           &= 2 + \log_2{n}
    \end{align*}

    So: 
    \begin{align*}
    f'(n) - f(n) &= (2 + \log_2{n}) - (\log_2{n}) \\
                 &= 2
    \end{align*}
    The function's value increases by precisely $2$. \\

  %-----------------   B   -----------------------------------------------------
    \item $ f(n)=\sqrt{n} $ \\

    \textbf{Solution:} Again, I will replace the $n$ in $f(n)$ with
    $4n$ to create $f'(n)$, but this time compute the \emph{ratio} of the two
    functions to find a \emph{multiplicative factor} to illustrate
    the change in value of the function.\\
    Let $f'(n) = \sqrt{4n}$

    \begin{align*}
    f'(4n) &= \sqrt{4n}            \\
           &= \sqrt{4} * \sqrt{n}  \\
           &= 2 * \sqrt{n}
    \end{align*}

    So: 
    \begin{align*}
    \frac{f'(n)}{f(n)} &= \frac{2\sqrt{n}}{\sqrt{n}} \\
                 &= 2
    \end{align*}
    The function's value increases by a factor of $2$. \\

  %-----------------   C   -----------------------------------------------------
    \item $ f(n)=n $ \\

    \textbf{Solution:} Again, I will replace the $n$ in $f(n)$ with
    $4n$ to create $f'(n)$ and compute the \emph{ratio} of the two
    functions to find a \emph{multiplicative factor} to illustrate
    the change in value of the function.\\
    Let $f'(n) = 4n$

    \begin{align*}
    f'(n) &= 4n            \\
    \end{align*}

    So: 
    \begin{align*}
    \frac{f'(n)}{f(n)} &= \frac{4}{n} \\
                 &= 4
    \end{align*}
    The function's value increases by a factor of $4$. \\

  %-----------------   D   -----------------------------------------------------
    \item $ f(n)=n^{3} $ \\

    \textbf{Solution:} Again, I will replace the $n$ in $f(n)$ with
    $4n$ to create $f'(n)$ and compute the \emph{ratio} of the two
    functions to find a \emph{multiplicative factor} to illustrate
    the change in value of the function.\\
    Let $f'(n) = (4n)^{3}$

    \begin{align*}
    f'(n) &= (4n)^{3}            \\
          &= 4^{3} * n^{3}       \\
          &= 64 * n^{3}          \\
          &= 64n^{3}
    \end{align*}

    So: 
    \begin{align*}
    \frac{f'(n)}{f(n)} &= \frac{64n^{3}}{n^{3}} \\
                       &= 64
    \end{align*}
    The function's value increases by a factor of $64$. \\

  %-----------------   E   -----------------------------------------------------
    \item $ f(n)=2^{n} $ \\

    \textbf{Solution:} Again, I will replace the $n$ in $f(n)$ with
    $4n$ to create $f'(n)$ and compute the \emph{ratio} of the two
    functions to find a \emph{multiplicative factor} to illustrate
    the change in value of the function.\\
    Let $f'(n) = 2^{4n}$

    \begin{align*}
    f'(n) &= 2^{4n}
    \end{align*}

    So: 
    \begin{align*}
    \frac{f'(n)}{f(n)} &= \frac{2^{4n}}{2^{n}} \\
                 &= 2^{4n-n}                   \\
                 &= 2^{3n}                     \\
                 &= (2^{3})^{n}                \\
                 &= 8^{n}
    \end{align*}
    The function's value changes by  factor of $ 8^{n} $. \\

    \end{enumerate}

\end{enumerate}

\end{document}
