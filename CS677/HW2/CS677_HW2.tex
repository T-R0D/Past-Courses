%===============================================================================
%=======     Document Information     ==========================================
%===============================================================================

\title{CS 677: Assignment 1}
\author{Terence Henriod}
\date{\today}
\documentclass[11pts]{article}


%===============================================================================
%=======     Packages Used     =================================================
%===============================================================================

%\usepackage{qtree}
\usepackage{amsmath}
\usepackage{amssymb}
\usepackage[named]{algo}
\usepackage{verbatim}
\usepackage[right=1in,top=1in,left=1in,bottom=1in]{geometry}


%===============================================================================
%=======     Document Settings     =============================================
%===============================================================================

\setlength{\topmargin}{-1cm}
\setlength{\oddsidemargin}{0in}
\setlength{\textwidth}{6.5in}
\setlength{\textheight}{8.3in}


%%Currently default settings for indentation and symbols.
%%Try these by uncommenting this block!!!
%%Redefine the first level symbols
%\renewcommand{\theenumi}{\fnsymbol{enumi}-}
%\renewcommand{\labelenumi}{\theenumi}
%
%%Redefine the second level symbols
%\renewcommand{\theenumii}{\alph{enumii})}
%\renewcommand{\labelenumii}{\theenumii}
%
%%Redefine the third level symbols
%\renewcommand{\theenumiii}{\roman{enumiii}.}
%\renewcommand{\labelenumiii}{\theenumiii}
%
%%Options for redefining levels


%\arabic
%\alph 
%\Alph
%\roman
%\Roman
%\fnsymbol
%This ^^^ is all you need to change!!


%===============================================================================
%=======     User Defined Commands     =========================================
%===============================================================================

\newcommand{\BigO}[1]{\ensuremath{\operatorname{O}\bigl(#1\bigr)}}
\newcommand{\BigTheta}[1]{\ensuremath{\operatorname{\Theta}\bigl(#1\bigr)}}
\newcommand{\BigOmega}[1]{\ensuremath{\operatorname{\Omega}\bigl(#1\bigr)}}


%===============================================================================
%=======     The Document     ==================================================
%===============================================================================
\begin{document}

\maketitle

%~~~~~~~~~~~~~~~~~~~~~~~~~~~~~~~~~~~~~~~~~~~~~~~~~~~~~~~~~~~~~~~~~~~~~~~~~~~~~~~
%~~~~~~~     Abstract     ~~~~~~~~~~~~~~~~~~~~~~~~~~~~~~~~~~~~~~~~~~~~~~~~~~~~~~
%~~~~~~~~~~~~~~~~~~~~~~~~~~~~~~~~~~~~~~~~~~~~~~~~~~~~~~~~~~~~~~~~~~~~~~~~~~~~~~~
\begin{abstract}
In this assignment, methods for solving recurrences are practiced. Methods
considered include iteration, substitution, the recursion tree, and the
"Master's" method.
\end{abstract}
\newpage

% begin the problem set
\begin{enumerate}
%~~~~~~~~~~~~~~~~~~~~~~~~~~~~~~~~~~~~~~~~~~~~~~~~~~~~~~~~~~~~~~~~~~~~~~~~~~~~~~~
%~~~~~~~     Number 1     ~~~~~~~~~~~~~~~~~~~~~~~~~~~~~~~~~~~~~~~~~~~~~~~~~~~~~~
%~~~~~~~~~~~~~~~~~~~~~~~~~~~~~~~~~~~~~~~~~~~~~~~~~~~~~~~~~~~~~~~~~~~~~~~~~~~~~~~
\item Consider the following algorithm:
\begin{algorithm}{Enigma}[ A( 0 \dots n-1, 0 \dots n-1 ) ]{
  \qcomment{// Input: a matrix A[0 \dots n-1, 0 \dots n-1] of integer numbers} }
\qfor{ $i \qlet 0$ \qto $n - 2$ } \\
\do \\
  \qfor{ $j \qlet i + 1$ \qto $n - 1 $ } \\
  \do \\
    \qif{ $ A[i, j] \ne A[j, i] $ } \\
      \qreturn \qfalse \qfi \qrof \qrof \\
\qreturn \qtrue
\end{algorithm}

  \begin{enumerate}
  %-----------------------------------------------------------------------------
  %-------     A     -----------------------------------------------------------
  %-----------------------------------------------------------------------------
  \item What does this algorithm do? \\

  \textbf{Solution:} This algorithm checks to see if all of the elements
  below the main diagonal of the matrix are equal to the element that
  is transposed from the
  original element in a matrix, i.e. symmetric about, but not including,
  the diagonal. If the matrix is symmetric in this manner, this is indicated
  by the returning of true after checking all the elements. If any pair of
  elements are detected to be asymmetric, the check halts and false is
  returned. \\

  %-----------------------------------------------------------------------------
  %-------     B     -----------------------------------------------------------
  %-----------------------------------------------------------------------------
  \item Compute the running time of this algorithm. \\

  \textbf{Solution:} Outline the number of times each primitive operation
  (primitive math operations, comparisons, etc.) occurs, consider primitive
  operations to have constant running time. Note that $t_x$ indicates the
  number of times a statement is executed for a given value of $x$. Then sum
  the cost multiples to get the running time of the algorithm. \\

  \textit{ Numbers refer to the steps of the algorithm above.}
    \begin{center}
    \begin{tabular}{| c || c | c | c |}
      \hline
      Step Number & Cost & Times Executed (Best Case)
          & Times Executed (Worst Case)  \\
      \hline \hline
      1.          & $c_1$  & $1$ & $\sum_{i=0}^{n-2}{1} = n - 1$  \\
      \hline
      2.          & $c_2$  & $1$ & $\sum_{i=0}^{n-2}{t_i} $ \\
      \hline
      3.          & $c_3$  & $1$ & $\sum_{i=0}^{n-2}{t_i - 1}$  \\
      \hline
      4.          & $c_4$  & $1$ & $0$  \\
      \hline
      5.          & $c_5$  & $0$ & $1$  \\
      \hline
    \end{tabular}
    \end{center}

  In the "best" case scenario, the very first pair of elements A[$i,j$] and
  A[$j,i$] are not equal, thus the first for-loop check will only execute once,
  the inner for-loop check will only execute once, and the if statement will
  only execute once, and of course the return statement will return once.
  In this case, $t_i = 1$, but this will make little difference overall:
  \begin{align*}
    T(n) &= c_1(1) + c_2(1) + c_3(1) + c_4(1) + c_5(0) \\
         &= c_1 + c_2 + c_3 + c_4 \\
         &= c_a \text{(an arbitrary constant)} \\
         &= \BigTheta{1}
  \end{align*}
%  In the "best" case, this function is \BigTheta{1}.
\newpage
  In the "worst" case, both loops will execute the maximum number of times
  because each pair of elements tested will be checked before true is
  finally returned.
  In this case the value for $t_i$ will be:
  \begin{align*}
    t_i &= \sum_{j=i+1}^{n-1}{1} \\
        &= ((n - 1) - (i + 1)) + 1 \\
        &= (n - i - 2) + 1 \\
        &= n - i - 1
  \end{align*}

  So the resulting time cost is:
  \begin{align*}
    T(n) &= c_1(\sum_{i=0}^{n-2}{1}) + c_2(\sum_{i=0}^{n-2}{t_i}) +
            c_3(\sum_{i=0}^{n-2}{t_i - 1}) + c_4(0) + c_5(1) \\
         &= c_1(n - 1) + c_2(\sum_{i=0}^{n-2}{n - i - 1}) +
            c_3(\sum_{i=0}^{n-2}{(n - i - 1) - 1}) + c_5 \\
         &= c_1(n - 1) +
            c_2(\sum_{i=0}^{n-2}{n}-\sum_{i=0}^{n-2}{i}-\sum_{i=0}^{n-2}{1}) +
            c_3(\sum_{i=0}^{n-2}{n - i - 2}) + c_5 \\
         &= c_1(n - 1) +
            c_2((n - 1)(n) - (\frac{(n-1)n}{2}) - (n - 1)) +
            c_3(\sum_{i=0}^{n-2}{n}-\sum_{i=0}^{n-2}{i}-\sum_{i=0}^{n-2}{2}) +
            c_5 \\
         &= c_1(n - 1) + c_2((n^{2} - n) - (\frac{n^{2} - n}{2}) - (n - 1)) +
            c_3((n - 1)(n) - (\frac{(n-1)n}{2}) - 2(n - 1)) + c_5 \\
         &= c_1(n - 1) + c_2(\frac{n^{2} - n}{2} - (n - 1)) +
            c_3(\frac{n^{2} - n}{2} - (2n - 2)) + c_5 \\
         &= c_1(n - 1) + c_2(\frac{n^{2} - 3n + 2}{2}) +
            c_3(\frac{n^{2} - 5n + 4}{2}) + c_5 \\
         &= an^{2} + bn + c \text{, for sufficient a, b, c} \\
         &= \BigTheta{n^{2}}
  \end{align*}
  \end{enumerate}
\newpage

%~~~~~~~~~~~~~~~~~~~~~~~~~~~~~~~~~~~~~~~~~~~~~~~~~~~~~~~~~~~~~~~~~~~~~~~~~~~~~~~
%~~~~~~~     Number 2     ~~~~~~~~~~~~~~~~~~~~~~~~~~~~~~~~~~~~~~~~~~~~~~~~~~~~~~
%~~~~~~~~~~~~~~~~~~~~~~~~~~~~~~~~~~~~~~~~~~~~~~~~~~~~~~~~~~~~~~~~~~~~~~~~~~~~~~~
\item Solve the following recurrences using a method of your choice:
  \begin{enumerate}
  %-----------------------------------------------------------------------------
  %-------     A     -----------------------------------------------------------
  %-----------------------------------------------------------------------------
  \item $ T(n) = 4T(\frac{n}{3}) + n^{2} $ \\

  \textbf{Solution:} One method to use is the "Master's Method." To use this
  method, we must put the expression into the form:
  $$ T(n) = aT(\frac{n}{b}) + f(n) $$
  and then, based on which of three cases $T(n)$ and $f(n)$ fall
  into, we can make our decision:
  \begin{align*}
  \text{Case 1: } & \text{if } f(n) = \BigO{n^{\log_{b}{a-\epsilon}}}
      \text{ for some } \epsilon > 0,
      \text{ then } T(n) = \BigTheta{n^{\log_{b}{a}}} \\
  \text{Case 2: } & \text{if } f(n) = \BigTheta{n^{\log_{b}{a}}}
      \text{ then } T(n) = \BigTheta{n^{\log_{b}{a}} \lg{n}} \\
  \text{Case 3: } & \text{if } f(n) = \BigOmega{n^{\log_{b}{a+\epsilon}}}
      \text{ for some } \epsilon > 0, \\
      & \text{ and if } a*f(\frac{n}{b}) \le c*f(n)
        \text{ for some } c < 1 \text{ and all sufficiently large n, } \\
      & \text{ then } T(n) = \BigTheta{f(n)}
  \end{align*}
  By applying the Master's method to this recurrence, we have
  $a=4$, $b=3$, and $f(n) = n^{2}$, giving:
  \[ T(n) = 4T(\frac{n}{3}) + n^{2} \rightarrow n^{\log_{3}{4}}\approx
        n^{1.26} \rightarrow f(n) = \BigOmega{n^{\log_{3}{4}}} \]
  Thus, the recurrence falls under Case 3. Since it falls under case 3,
  we must ensure that the regularity condition is met:
  \begin{align*}
       a*f(\frac{n}{b}) &\le c*f(n) \text{, for all $c > 1$} \\
    (4)f(\frac{n}{(3)}) &\le c*f(n) \\
    4*(\frac{n}{3})^{2} &\le c*(n)^{2} \\
     4(\frac{n^{2}}{9}) &\le cn^{2} \\
       \frac{4}{9}n^{2} &\le cn^{2}
           \text{, }\forall c \text{ s.t. }\frac{4}{9} \le c < 1
  \end{align*}
  The regularity condition holds, so: $ T(n) = \BigTheta{n^{2}} $
  \newline
  \newpage
  %-----------------------------------------------------------------------------
  %-------     B     -----------------------------------------------------------
  %-----------------------------------------------------------------------------
  \item $ T(n) = T(n - 1) + 5 $ \\

  \textbf{Solution:} Using the "tree" method, we find out how many times
  the recurrence will execute before reaching its base case (problem size of
  1), and count the cost of each recursive execution. \\

  In this problem, there isn't really much of a "tree" since the recurunce
  only takes one path, but we can still use the method.

  The cost of one execution is $5 + T(n-1)$, which is a cost of $5$ per
  recurrence execution plus a constant.

  Next we must find how many times $i$ the recurrence will execute. Note
  that each time the recurrence executes, the problem size decreases by one
  until the problem size is one. So the total number of times $i$ the
  recurrence will execute is:
  \begin{align*}
    n - 1i &= 1 \\
    -i     &= 1 - n \\
    i      &= n - 1 
  \end{align*}

  Assuming that the running time of $T(1)$/$T(0)$ is constant, putting it
  all together, we get:
  \[ T(n) = 5i = 5(n-1) = \BigTheta{n} \]

  \end{enumerate}
\newpage



%~~~~~~~~~~~~~~~~~~~~~~~~~~~~~~~~~~~~~~~~~~~~~~~~~~~~~~~~~~~~~~~~~~~~~~~~~~~~~~~
%~~~~~~~     Number 3     ~~~~~~~~~~~~~~~~~~~~~~~~~~~~~~~~~~~~~~~~~~~~~~~~~~~~~~
%~~~~~~~~~~~~~~~~~~~~~~~~~~~~~~~~~~~~~~~~~~~~~~~~~~~~~~~~~~~~~~~~~~~~~~~~~~~~~~~
\item Consider the following recursive algorithm for computing the sum
of the first $n$ cubes:
\[ S(n) = 1^{3} + 2^{3} + \ldots + n^{3} \]
\begin{algorithm}{S}[ n ]{
  \qcomment{// Input: A positive integer $n$}
  \qcomment{// Output: The sum of the first n cubes} }
\qif{$n = 1$} \\
  \qreturn $1$ \\
\qelse \\
  \qreturn $ S(n - 1) + (n * n * n) $ \qfi
\end{algorithm}

  \begin{enumerate}
  %-----------------------------------------------------------------------------
  %-------     A     -----------------------------------------------------------
  %-----------------------------------------------------------------------------
  \item Write and solve a recurrence relation for the number of
  multiplications made by this algorithm and solve it. \\

  \textbf{Solution:} Assuming that multiplications are primitive operations,
  we have \[ S(n) = S(n - 1) + 4 \] because the "problem size" decreases
  by one with each recursive call, and the cost incurred by each
  recursive call (other than the next recursive call) is one comparison,
  two multiplications and one addition. \\

  %-----------------------------------------------------------------------------
  %-------     B     -----------------------------------------------------------
  %-----------------------------------------------------------------------------
  \item How does this algorithm compare with the straightforward
  non-recursive algorithm for computing this function? \\

  \textbf{Solution:} Using the "tree" method, we can determine the running
  time for the recursive method. It clearly has a cost of 4 per recursive
  call, except for in the base case where it only has a cost of 1 comparison
  and the return of 1, i.e. a constant cost per recursice call. Since
  the "problem size" is only reduced by one with each call, and assuming that
  it will take $i$ calls to solve the problem, we find that:
  \begin{align*}
    n - 1i &= 1 \\
    -i     &= 1 - n \\
    i      &= n - 1 
  \end{align*}
  $n - 1$ calls are needed to resolve the problem. Thus, the recursive
  algorithm has running time
  \[ S(n) = ci = c(n - 1) = \BigTheta{n} \]

  As for the iterative algorithm, it would look like:
  \begin{algorithm}{S}[ n ]{
    \qcomment{// Input: A positive integer $n$}
    \qcomment{// Output: The sum of the first n cubes} }
  \qfor{$ S \qlet 0, i \qlet 0 \qto n $} \\
    \qdo{ $ S \qlet S + i^{3} $ }\qrof
  \qreturn $ S $
  \end{algorithm}

  Simple cost analysis would indicate that the loop runs $n + 1$ times,
  and the actions inside the loop run $n$ times, giving
  \[ S(n) = c_1(n + 1) + c_2(n) + c_3(1) = an + b = \BigTheta{n} 
      \text{, for sufficient $a$ and $b$ } \]

  Thus the running times are very comparable. In practice, the iterative
  algorithm would actually run faster due to the lack of overhead
  associated with recursive calls, but the two algorithms do have the same
  order of growth.

  \end{enumerate}

\newpage

%~~~~~~~~~~~~~~~~~~~~~~~~~~~~~~~~~~~~~~~~~~~~~~~~~~~~~~~~~~~~~~~~~~~~~~~~~~~~~~~
%~~~~~~~     Number 4     ~~~~~~~~~~~~~~~~~~~~~~~~~~~~~~~~~~~~~~~~~~~~~~~~~~~~~~
%~~~~~~~~~~~~~~~~~~~~~~~~~~~~~~~~~~~~~~~~~~~~~~~~~~~~~~~~~~~~~~~~~~~~~~~~~~~~~~~
\item Consider the following recursive algorithm:
\begin{algorithm}{Min}[ A, l, r ]{
  \qcomment{// Input: An array A[0 \dots n-1] of integer numbers}
  \qcomment{// The initial call is \textit{Min( A, 0, n-1)}} }
\qif{$l = r$} \\
  \qreturn $A[l]$ \\
\qelse{$temp1 \qlet Min(A, l, \lfloor \frac{l + r}{2} \rfloor )$} \\
       $temp2 \qlet Min(A, \lfloor \frac{l + r}{2} \rfloor + 1, r)$ \\
  \qif{$temp1 \le temp2$} \\
    \qreturn $temp1$ \\
  \qelse \\
    \qreturn $temp2$ \qfi \qfi 
\end{algorithm}

  \begin{enumerate}
  %-----------------------------------------------------------------------------
  %-------     A     -----------------------------------------------------------
  %-----------------------------------------------------------------------------
  \item Write the recurrence relation for the above algorithm. \\

  \textbf{Solution:} One might write the recurrence relation as follows:
  \[ M(n) = 2M(\frac{n}{2}) + c \]
  because does make some comparisons and a return with each recursive call,
  but we can also see that when further recursive calls are made, the problem
  size is halved by moving one of the indices to the median value of the
  array given for the current recursive call and passing the smaller array
  to the next recursive call. Each half of the given array is given to one
  of the two recursive calls. It should be noted that all calls to the 
  function/algorithm will generate sub-trees of equal height/size, but it
  can be reasonably said that the actual cost of the algorithm will be less
  than or equal to the running time that is found by assuming all recursive
  calls generate subtrees of precisely the same height/size. It should be
  noted that these "sub-tree" imbalances occur when integer division of an
  odd number creates a pair of unequal halves. \\

  %-----------------------------------------------------------------------------
  %-------     B     -----------------------------------------------------------
  %-----------------------------------------------------------------------------
  \item Solve the recurrence obtained in part (a). \\

  \textbf{Solution:} Using the tree method (appropriately this time), we
  can solve the running time of this recurrence. The cost associated with
  each recursive call is a constant plus two recursive calls (with the
  exception of base cases, which have only a constant cost). Then we
  need to compute the number of times the algorithm will run. This number,
  $i$, is found by solving the following equation: 
  \begin{align*}
    \frac{n}{2^{i}} &= 1 \\
    2^{i}           &= n \\
    i               &= \lg{n}
  \end{align*}
  So, to compute the total cost/running time of the algorithm, we have:
  \[ M(n) = ci = c(\lg{n}) = \BigTheta{\lg{n}} \]

  \end{enumerate}
\newpage

%~~~~~~~~~~~~~~~~~~~~~~~~~~~~~~~~~~~~~~~~~~~~~~~~~~~~~~~~~~~~~~~~~~~~~~~~~~~~~~~
%~~~~~~~     Number 5     ~~~~~~~~~~~~~~~~~~~~~~~~~~~~~~~~~~~~~~~~~~~~~~~~~~~~~~
%~~~~~~~~~~~~~~~~~~~~~~~~~~~~~~~~~~~~~~~~~~~~~~~~~~~~~~~~~~~~~~~~~~~~~~~~~~~~~~~
\item Consider the following algorithm:
\begin{algorithm}{Mystery}[ n ]{
  \qcomment{// Input: A nonnegative integer n} }
$S \qlet 0$ \\
\qfor{$i \qlet 1$ \qto $n$} \\
\qdo \\
  $S \qlet S + i * i$ \qrof \\
\qreturn $S$
\end{algorithm}

  \begin{enumerate}
  %-----------------------------------------------------------------------------
  %-------     A     -----------------------------------------------------------
  %-----------------------------------------------------------------------------
  \item What does this algorithm compute? \\

  \textbf{Solution:} This algorithm computes the sum of all the squares
  with roots $1$ to $n$. Every number from $1$ to $n$ is visited (step 2), each
  of those numbers are squared (step 4), and the resulting square is added
  to the running total (step 4). \\ %Prove by induction?

  %-----------------------------------------------------------------------------
  %-------     B     -----------------------------------------------------------
  %-----------------------------------------------------------------------------
  \item Compute the running time of this algorithm. \\

  \textbf{Solution:} We can begin a cost anaysis by laying out the cost and
  frequency of each step in a table:
    \begin{center}
    \begin{tabular}{| c || c | c |}
      \hline
      Step Number & Cost & Times Executed  \\
      \hline \hline
      1.          & $c_1$ & $1$ \\
      \hline
      2.          & $c_2$ & $n + 1$ \\
      \hline
      3.          & ---   & --- \\
      \hline
      4.          & $c_3$ & $n * 3$ \\
      \hline
      5.          & $c_4$ & $1$  \\
      \hline
    \end{tabular}
    \end{center}
  Which, using the values in the table to compute the running time, gives:
  \begin{align*}
    T(n) &= c_1(1) + c_2(n + 1) + c_3(3n) + c_4(1) \\
         &= an + b \\
         &= \BigTheta{n}
  \end{align*}

  \end{enumerate
%\LAST PAGE
\end{enumerate}

\end{document}

